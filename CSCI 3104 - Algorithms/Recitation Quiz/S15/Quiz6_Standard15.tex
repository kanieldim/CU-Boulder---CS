\documentclass[11pt]{article}
\setlength{\oddsidemargin}{0in}
\setlength{\evensidemargin}{0in}
\setlength{\textwidth}{6.5in}
\setlength{\parindent}{0in}
\setlength{\parskip}{\baselineskip}
\usepackage{amsmath,amsfonts,amssymb}
\usepackage{enumitem}
\usepackage{graphicx}
\usepackage[]{algorithmicx}
\usepackage{comment}
\usepackage{tkz-berge}
\usetikzlibrary{positioning, automata}

\usepackage{fancyhdr}
\pagestyle{fancy}
\setlength{\headsep}{36pt}

\usepackage{hyperref}



\newcommand{\makenonemptybox}[2]{%
%\par\nobreak\vspace{\ht\strutbox}\noindent
\item[]
\fbox{% added -2\fboxrule to specified width to avoid overfull hboxes
% and removed the -2\fboxsep from height specification (image not updated)
% because in MWE 2cm is should be height of contents excluding sep and frame
\parbox[c][#1][t]{\dimexpr\linewidth-2\fboxsep-2\fboxrule}{
  \hrule width \hsize height 0pt
  #2
 }%
}%
\par\vspace{\ht\strutbox}
}
\makeatother

\begin{document}
\definecolor {processblue}{cmyk}{0.96,0,0,0}
\lhead{{\bf CSCI 3104, Algorithms \\ Quiz 6 } }
\rhead{Name: \fbox{\phantom{This is a really long name}} \\ ID: \fbox{\phantom{This is a student ID}} \\ {\bf Profs.\ Chen \& Grochow\\ Spring 2020, CU-Boulder}}
\renewcommand{\headrulewidth}{0.5pt}

\phantom{Test}

\begin{small}
\noindent \textbf{Instructions:} This quiz is open book and open note, but \textbf{not} open-internet. You \textbf{may} post clarification questions to Piazza, with the understanding that you may not receive an answer in time and posting does count towards your 30 minutes. Questions posted to Piazza \textbf{must be posted as PRIVATE QUESTIONS.} Other use of the internet, including searching for answers or posting to sites like Chegg, is strictly prohibited. Any violation of the honor code is grounds to receive a 0 on this quiz. Proofs should be written in \textbf{complete sentences.} \textbf{Show and justify all work to receive full credit.}
\end{small} 


\hrulefill 

	\noindent \textbf{Standard 15.} Consider the following directed, weighted graph $G$. At the first iteration of Dijksrta's Algorithm, using $A$ as the source vertex, we examine both the $(A, B)$ and $(A, C)$ edges by placing them into a priority queue. However, only $(A, B)$ is selected at the first iteration. 

\begin{center}
	\begin {tikzpicture}[-latex ,auto ,node distance =2 cm and 3cm ,on grid ,
	semithick ,
	state/.style ={ circle ,top color =white , bottom color = processblue!20 ,
	draw,processblue , text=blue , minimum width =1 cm}, scale=0.5]

	\node[state] (A) {$A$};
	\node[state] (B) [above right = of A] {$B$};
	\node[state] (C) [below right = of A] {$C$};
	\node[state] (G) [below right = of B] {$G$};
	\node[state] (D) [above right = of G] {$D$};
	\node[state] (E) [below right = of G] {$E$};
	\node[state] (F) [below right = of D] {$F$};
	
	\path (A) edge node[above] {$4$} (B);
	\path (A) edge node[above] {$8$} (C);
	\path (B) edge node[above] {$1$} (G);
	\path (B) edge node[above] {$6$} (D);
	\path (G) edge node[above] {$2$} (A);
	\path (E) edge node[above] {$1$} (C);
	\path (D) edge node[above] {$3$} (G);
	\path (D) edge node[above] {$10$} (F);
	\path (F) edge node[above] {$2$} (E);
	\path (G) edge node[above] {$5$} (E);
	\path (G) edge node[above] {$2$} (F);
	\path (G) edge node[above] {$2$} (C);
	
	\end{tikzpicture}  
\end{center}

\noindent 

What are the next five edges \textbf{selected} by Dijkstra's algorithm? After these have been selected, what are the distances from $A$ that the algorithm has recorded for each vertex in $G$? 

% YOUR ANSWER HERE

\end{document}


